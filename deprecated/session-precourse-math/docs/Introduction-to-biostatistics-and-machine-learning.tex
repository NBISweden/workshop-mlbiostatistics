% Options for packages loaded elsewhere
\PassOptionsToPackage{unicode}{hyperref}
\PassOptionsToPackage{hyphens}{url}
%
\documentclass[
]{book}
\usepackage{amsmath,amssymb}
\usepackage{lmodern}
\usepackage{iftex}
\ifPDFTeX
  \usepackage[T1]{fontenc}
  \usepackage[utf8]{inputenc}
  \usepackage{textcomp} % provide euro and other symbols
\else % if luatex or xetex
  \usepackage{unicode-math}
  \defaultfontfeatures{Scale=MatchLowercase}
  \defaultfontfeatures[\rmfamily]{Ligatures=TeX,Scale=1}
\fi
% Use upquote if available, for straight quotes in verbatim environments
\IfFileExists{upquote.sty}{\usepackage{upquote}}{}
\IfFileExists{microtype.sty}{% use microtype if available
  \usepackage[]{microtype}
  \UseMicrotypeSet[protrusion]{basicmath} % disable protrusion for tt fonts
}{}
\makeatletter
\@ifundefined{KOMAClassName}{% if non-KOMA class
  \IfFileExists{parskip.sty}{%
    \usepackage{parskip}
  }{% else
    \setlength{\parindent}{0pt}
    \setlength{\parskip}{6pt plus 2pt minus 1pt}}
}{% if KOMA class
  \KOMAoptions{parskip=half}}
\makeatother
\usepackage{xcolor}
\usepackage{longtable,booktabs,array}
\usepackage{calc} % for calculating minipage widths
% Correct order of tables after \paragraph or \subparagraph
\usepackage{etoolbox}
\makeatletter
\patchcmd\longtable{\par}{\if@noskipsec\mbox{}\fi\par}{}{}
\makeatother
% Allow footnotes in longtable head/foot
\IfFileExists{footnotehyper.sty}{\usepackage{footnotehyper}}{\usepackage{footnote}}
\makesavenoteenv{longtable}
\usepackage{graphicx}
\makeatletter
\def\maxwidth{\ifdim\Gin@nat@width>\linewidth\linewidth\else\Gin@nat@width\fi}
\def\maxheight{\ifdim\Gin@nat@height>\textheight\textheight\else\Gin@nat@height\fi}
\makeatother
% Scale images if necessary, so that they will not overflow the page
% margins by default, and it is still possible to overwrite the defaults
% using explicit options in \includegraphics[width, height, ...]{}
\setkeys{Gin}{width=\maxwidth,height=\maxheight,keepaspectratio}
% Set default figure placement to htbp
\makeatletter
\def\fps@figure{htbp}
\makeatother
\setlength{\emergencystretch}{3em} % prevent overfull lines
\providecommand{\tightlist}{%
  \setlength{\itemsep}{0pt}\setlength{\parskip}{0pt}}
\setcounter{secnumdepth}{5}
\usepackage[swedish,english]{babel}
\ifLuaTeX
  \usepackage{selnolig}  % disable illegal ligatures
\fi
\IfFileExists{bookmark.sty}{\usepackage{bookmark}}{\usepackage{hyperref}}
\IfFileExists{xurl.sty}{\usepackage{xurl}}{} % add URL line breaks if available
\urlstyle{same} % disable monospaced font for URLs
\hypersetup{
  pdftitle={Mathematical foundations},
  pdfauthor={Olga Dethlefsen},
  hidelinks,
  pdfcreator={LaTeX via pandoc}}

\title{Mathematical foundations}
\author{Olga Dethlefsen}
\date{2022-08-30}

\usepackage{amsthm}
\newtheorem{theorem}{Theorem}[chapter]
\newtheorem{lemma}{Lemma}[chapter]
\newtheorem{corollary}{Corollary}[chapter]
\newtheorem{proposition}{Proposition}[chapter]
\newtheorem{conjecture}{Conjecture}[chapter]
\theoremstyle{definition}
\newtheorem{definition}{Definition}[chapter]
\theoremstyle{definition}
\newtheorem{example}{Example}[chapter]
\theoremstyle{definition}
\newtheorem{exercise}{Exercise}[chapter]
\theoremstyle{remark}
\newtheorem*{remark}{Remark}
\newtheorem*{solution}{Solution}
\begin{document}
\maketitle

{
\setcounter{tocdepth}{1}
\tableofcontents
}
\hypertarget{preface}{%
\chapter*{Preface}\label{preface}}
\addcontentsline{toc}{chapter}{Preface}

Biostatistics and machine learning is based on mathematics. This precourse is here to remind you some important notations and conventions used.

Do you see a mistake or a typo? I would be grateful if you let me know via \url{olga.dethlefsen@nbis.se}

\emph{This repository contains teaching and learning materials prepared and used during ``Introduction to biostatistics and machine learning'' course, organized by NBIS, National Bioinformatics Infrastructure Sweden. The course is open for PhD students, postdoctoral researcher and other employees within Swedish universities. The materials are geared towards life scientists wanting to be able to understand and use basic statistical and machine learning methods. More about the course \url{https://nbisweden.github.io/workshop-mlbiostatistics/}}

\hypertarget{acknowledgments}{%
\chapter*{Acknowledgments}\label{acknowledgments}}
\addcontentsline{toc}{chapter}{Acknowledgments}

Thank you to everyone helping out to make these materials better. A warm thank you goes to:
Eva Freyhult, Payam Emami, Mun-Gwan Hong, Nima Rafati, Cecilia Hellstr\(\ddot o\)m, Oliver Konzock, Franziska Hildebrandt.

\hypertarget{mathematical-notations}{%
\chapter{Mathematical notations}\label{mathematical-notations}}

\textbf{Aims}

\begin{itemize}
\tightlist
\item
  to recapitulate the basic notations and conventions used in mathematics and statistics
\end{itemize}

\textbf{Learning outcomes}

\begin{itemize}
\tightlist
\item
  to recognize natural numbers, integers and real numbers
\item
  to understand the differences between variables and constants
\item
  to use symbols, especially Sigma and product notations, to represent mathematical operations
\end{itemize}

\hypertarget{numbers}{%
\section{Numbers}\label{numbers}}

\begin{itemize}
\tightlist
\item
  \textbf{Natural numbers, N}: numbers such as 0, 1, 3, \ldots{}
\item
  \textbf{Integers, Z}: include negative numbers \ldots, -2, -1, 0, 1, 2
\item
  \textbf{Rational numbers}: numbers that can be expressed as a ratio two integers, i.e.~in a form \(\frac{a}{b}\), where \(a\) and \(b\) are integers, and \(b\neq0\)
\item
  \textbf{Real numbers, R}: include both rational and irrational numbers
\item
  \textbf{Reciprocal} of any number is found by diving 1 by the number, e.g.~reciprocal of 5 is \(\frac{1}{5}\)
\item
  \textbf{Absolute value} of a number can be viewed as its distance from zero, e.g.~the absolute value of 6 is 6, written as \(|6| = 6\) and absolute value of -5 is 5, written as \(|-5| = 5\)
\item
  \textbf{Factorial} of a non-negative integer number \(n\) is denoted by \(n!\) and it is a product of all positive integers less than or equal to \(n\), e.g.~\(4! = 4 \cdot 3\cdot 2 \cdot 1 = 24\)
\end{itemize}

\hypertarget{variables-constants-and-letters}{%
\section{Variables, constants and letters}\label{variables-constants-and-letters}}

Mathematics gives us a precise language to communicate different concepts and ideas. To be able to use it it is essential to learn symbols and understand how they are used to represent physical quantities as well as understand the rules and conventions that have been developed to manipulate them.

\begin{itemize}
\tightlist
\item
  \textbf{variables}: things that can vary, e.g.~temperature and time
\item
  \textbf{constants}: fixed and unchanging quantities used in certain calculations, e.g.~3.14159
\item
  in principle one could freely choose letters and symbols to represent variables and constants, but it is helpful and choose letters and symbols that have meaning in a particular context. Hence, we
\item
  \(x, y, z\), the end of the alphabet is reserved for variables
\item
  \(a, b, c\), the beginning of the alphabet is used to represent constants
\item
  \(\pi\), \(\omega\) and Greek letters below are used frequently used to represent common constant, e.g.~\(\pi = 3.14159\)
\end{itemize}

\begin{longtable}[]{@{}llllll@{}}
\caption{\label{tab:greek-table} Uppercase and lowercase letters of the Greek alphabet}\tabularnewline
\toprule()
Letter & Upper case & Lower case & Letter & Upper case & Lower case \\
\midrule()
\endfirsthead
\toprule()
Letter & Upper case & Lower case & Letter & Upper case & Lower case \\
\midrule()
\endhead
Alpha & A & \(\alpha\) & Nu & N & \(\nu\) \\
Beta & B & \(\beta\) & Xi & \(\Xi\) & \(\xi\) \\
Gamma & \(\Gamma\) & \(\gamma\) & Omicron & O & o \\
Delta & \(\Delta\) & \(\delta\) & Pi & \(\Pi\) & \(\pi\) \\
Epsilon & E & \(\epsilon\) & Rho & P & \(\rho\) \\
Zeta & Z & \(\zeta\) & Sigma & \(\Sigma\) & \(\sigma\) \\
Eta & H & \(\eta\) & Tau & T & \(\tau\) \\
Theta & \(\Theta\) & \(\theta\) & Upsilon & Y & \(\upsilon\) \\
Iota & I & \(\iota\) & Phi & \(\Phi\) & \(\phi\) \\
Kappa & K & \(\kappa\) & Chi & X & \(\chi\) \\
Lambda & \(\Lambda\) & \(\lambda\) & Psi & \(\Psi\) & \(\psi\) \\
Mu & M & \(\mu\) & Omega & \(\Omega\) & \(\omega\) \\
\bottomrule()
\end{longtable}

\hypertarget{a-precise-language}{%
\section{A precise language}\label{a-precise-language}}

\begin{itemize}
\tightlist
\item
  Mathematics is a precise language meaning that a special attention has to be paid to the exact position of any symbol in relation to other.
\item
  Given two symbols \(x\) and \(y\), \(xy\) and \(x^y\) and \(x_y\) can mean different things
\item
  \(xy\) stands for multiplication, \(x^y\) for superscript and \(x_y\) for subscript
\end{itemize}

\hypertarget{using-symbols}{%
\section{Using symbols}\label{using-symbols}}

If the letters \(x\) and \(y\) represent two numbers, then:

\begin{itemize}
\tightlist
\item
  their \textbf{sum} is written as \(x + y\)
\item
  subtracting \(y\) from \(x\) is \(x - y\), known also as \textbf{difference}
\item
  to multiply \(x\) and \(y\) we written as \(x \cdot y\) or also with the multiplication signed omitted as \(xy\). The quantity is known as \textbf{product of x and y}
\item
  multiplication is \textbf{associative}, e.g.~when we multiply three numbers together, \(x \cdot y \cdot z\), the order of multiplication does not matter, so \(x \cdot y \cdot z\) is the same as \(z \cdot x \cdot y\) or \(y \cdot z \cdot x\)
\item
  division is denoted by \(\frac{x}{y}\) and means that \(x\) is divided by \(y\). In this expression \(x\), on the top, is called \textbf{numerator} and \(y\), on the bottom, is called \textbf{denominator}
\item
  division by 1 leaves any number unchanged, e.g.~\(\frac{x}{1}=x\) and division by 0 is not allowed
\end{itemize}

Equal sign

\begin{itemize}
\tightlist
\item
  the equal sign \(=\) is used in \textbf{equations}, e.g.~\(x - 5 = 0\) or \(5x = 1\)
\item
  the equal sign \(=\) can be also used in \textbf{formulae}. Physical quantities are related through a formula in many fields, e.g.~the formula \(A=\pi r^2\) relates circle area \(A\) to its radius \(r\) and the formula \(s = \frac{d}{t}\) defines speed as distance \(d\) divided by time \(t\)
\item
  the equal sign \(=\) is also used in identities, expressions true for all values of the variable, e.g.~\((x-1)(x-1) = (x^2-1)\)
\item
  opposite to the equal sign is ``is not equal to'' sign \(\neq\), e.g.~we can write \(1+2 \neq 4\)
\end{itemize}

\textbf{Sigma and Product notation}

\begin{itemize}
\tightlist
\item
  the \(\Sigma\) notation, read as \textbf{Sigma notation}, provides a convenient way of writing long sums, e.g.~the sum of \(x_1 + x_2 + x_3 + ... + x_{20}\) is written as \(\displaystyle \sum_{i=1}^{i=20}x_i\)
\item
  the \(\Pi\) notation, read as \textbf{Product notation}, provides a convenient way of writing long products, e.g.~\(x_1 \cdot x_2 \cdot x_3 \cdot ... \cdot x_{20}\) is written as \(\displaystyle \prod_{i=1}^{i=20}x_i\)
\end{itemize}

\hypertarget{inequalities}{%
\section{Inequalities}\label{inequalities}}

Given any two real numbers \(a\) and \(b\) there are three mutually exclusive possibilities:

\begin{itemize}
\tightlist
\item
  \(a > b\), meaning that \(a\) is greater than \(b\)
\item
  \(a < b\), meaning that \(a\) is less than \(b\)
\item
  \(a = b\), meaning that \(a\) is equal to \(b\)
\end{itemize}

Strict and weak

\begin{itemize}
\tightlist
\item
  inequality in \(a > b\) and \(a < b\) is \textbf{strict}
\item
  as oppose to \textbf{weak} inequality denoted as \(a \ge b\) or \(a \le b\)
\end{itemize}

Some useful relations are:

\begin{itemize}
\tightlist
\item
  if \(a > b\) and \(b > c\) then \(a > c\)
\item
  if \(a > b\) then \(a + c > b\) for any positive \(c\)
\item
  if \(a > b\) then \(ac > bc\) for any positive \(c\)
\item
  if \(a > b\) then \(ac < bc\) for any negative \(c\)
\end{itemize}

\hypertarget{indices-and-powers}{%
\section{Indices and powers}\label{indices-and-powers}}

\begin{itemize}
\tightlist
\item
  \textbf{Indices}, also known as \textbf{powers} are convenient when we multiply a number by itself several times
\item
  e.g.~\(5 \cdot 5 \cdot 5\) is written as \(5^3\) and \(4 \cdot 4 \cdot 4 \cdot 4 \cdot 4\) is written as \(4^5\)
\item
  in the expression \(x^y\), \(x\) is called the \emph{base} and \(y\) is called \emph{index} or \emph{power}
\end{itemize}

The laws of indices state:

\begin{itemize}
\tightlist
\item
  \(a^m \cdot a^n = a^{m+n}\)
\item
  \(\frac{a^m}{a^n} = a^{m-n}\)
\item
  \((a^m)^n = a^{m\cdot n}\)
\end{itemize}

Rules derived from the laws of indices:

\begin{itemize}
\tightlist
\item
  \(a^0 = 1\)
\item
  \(a^1 = a\)
\end{itemize}

Negative and fractional indices:

\begin{itemize}
\tightlist
\item
  \(a^{-m} = \frac{1}{a^m}\) e.g.~\(5^{-2} = \frac{1}{5^2} = \frac{1}{25}\) for negative indices
\item
  e.g.~\(4^{\frac{1}{2}} = \sqrt{4}\) or \(8^{\frac{1}{3}} = \sqrt[3]{8}\) for fractional indices
\end{itemize}

\begin{center}\rule{0.5\linewidth}{0.5pt}\end{center}

\hypertarget{exercises-notations}{%
\section{Exercises: notations}\label{exercises-notations}}

\begin{exercise}
\protect\hypertarget{exr:m-notations-numbers}{}{\label{exr:m-notations-numbers} }
Classify numbers as natural, integers or real. If reall, specify if they are rational or irrational.

\begin{enumerate}
\def\labelenumi{\alph{enumi})}
\tightlist
\item
  \(\frac{1}{3}\)
\item
  2
\item
  \(\sqrt{4}\)
\item
  2.3
\item
  \(\pi\)
\item
  \(\sqrt{5}\)
\item
  -7
\item
  0
\item
  0.25
\end{enumerate}
\end{exercise}

\begin{exercise}
\protect\hypertarget{exr:m-notations-variables-constants}{}{\label{exr:m-notations-variables-constants} }Classify below descriptors as variables or constants. Do you know the letters or symbols commonly used to represent these?

\begin{enumerate}
\def\labelenumi{\alph{enumi})}
\tightlist
\item
  speed of light in vacuum
\item
  mass of an apple
\item
  volume of an apple
\item
  concentration of vitamin C in an apple
\item
  distance from Stockholm central station to Uppsala central station
\item
  time on the train to travel between the above stations
\item
  electron charge
\end{enumerate}
\end{exercise}

\begin{exercise}
\protect\hypertarget{exr:m-notations-sigma-product}{}{\label{exr:m-notations-sigma-product} }Write out explicitly what is meant by the following:

\begin{enumerate}
\def\labelenumi{\alph{enumi})}
\item
  \(\displaystyle \sum_{i=1}^{i=6}k_i\)
\item
  \(\displaystyle \prod_{i=1}^{i=6}k_i\)
\item
  \(\displaystyle \sum_{i=1}^{i=6}i^k\)
\item
  \(\displaystyle \prod_{i=1}^{i=3}i^k\)
\item
  \(\displaystyle \sum_{i=1}^{n}i\)
\item
  \(\displaystyle \sum_{i=1}^{i=4}(i + 1)^k\)
\item
  \(\displaystyle \prod_{i=1}^{i=4}(k + 1)^i\)
\item
  \(\displaystyle \prod_{i=0}^{n}i\)
\end{enumerate}
\end{exercise}

\begin{exercise}
\protect\hypertarget{exr:m-notations-sigma-product-reverse}{}{\label{exr:m-notations-sigma-product-reverse} }
Use Sigma or Product notation to represent the long sums and products below:

\begin{enumerate}
\def\labelenumi{\alph{enumi})}
\tightlist
\item
  \(1+2+3+4+5+6\)
\item
  \(2^2+3^2+4^2+5^2\)
\item
  \(4 \cdot 5 \cdot 6 \cdot 7 \cdot 8\)
\item
  \(1 + \frac{1}{2} + \frac{1}{3} + \frac{1}{4} + \frac{1}{5} +...+ \frac{1}{n}\)
\item
  \(2-2^2+2^3-2^4 + ...+2^n\)
\item
  \(3+6+9+12+···+60\)
\item
  \(3x + 6x^2 + 9x^3 + 12x^4 +...+60x^{20}\)
\item
  \(3x \cdot 6x^2 \cdot 9x^3 \cdot 12x^4 \cdot...\cdot 60x^{20}\)
\end{enumerate}
\end{exercise}

\hypertarget{answers-to-selected-exercises-notations}{%
\section*{Answers to selected exercises (notations)}\label{answers-to-selected-exercises-notations}}
\addcontentsline{toc}{section}{Answers to selected exercises (notations)}

Exr. \ref{exr:m-notations-numbers}

\begin{enumerate}
\def\labelenumi{\alph{enumi})}
\tightlist
\item
  real, rational
\item
  natural and integers, integers include natural numbers
\item
  \(\sqrt{4} = 2\) so it is a natural number and/subset of integers
\item
  real number, rational as it could be written as \(\frac{23}{10}\)
\item
  real number, irrational as it cannot be explained by a simple fraction
\item
  real number, irrational as it cannot be explained by a simple fraction
\item
  integer, non a natural number as these do not include negative numbers
\item
  natural number, although there is some argument about it as some define natural numbers as positive integers starting from 1, 2 etc. while others include 0.
\item
  real, rational number, could be written as \(\frac{25}{100}\)
\end{enumerate}

Exr. \ref{exr:m-notations-variables-constants}

\begin{enumerate}
\def\labelenumi{\alph{enumi})}
\tightlist
\item
  constant, speed of light in vacuum is a constant, denoted \(c\) with \(c=299 792 458 \frac{m}{s}\)
\item
  variable, mass of an apple is a variable, different for different apple sizes, for instance 138 grams, denoted as \(m = 100 g\)
\item
  variable, like mass volume can be different from apple to apple, denoted as \(V\), e.g.~\(V = 200 cm^3\)
\item
  variable, like volume and mass can vary, denoted as \(\rho_i\) and defined as \(\rho_i=\frac{m}{V}\). So given 6.3 milligrams of vitamin C in our example apple, we have \(\rho_i=\frac{0.0063}{2}\frac{g}{cm^3} = 0.0000315 \frac{g}{cm^3}\) concentration of vitamin D
\item
  constant, the distance between Stockholm and Uppsala is fixed; it could be a variable though if we were to consider an experiment on a very long time scale; distance is often denoted in physics as \(d\)
\item
  variable, time on the train to travel between the stations varies, often denoted as \(t\) with speed being calculated as \(s = \frac{d}{t}\)
\item
  constant, electron charge is \(e = 1.60217663\cdot10^{-19} C\)
\end{enumerate}

Exr. \ref{exr:m-notations-sigma-product}

\begin{enumerate}
\def\labelenumi{\alph{enumi})}
\item
  \(\displaystyle \sum_{i=1}^{i=6}k_i = k_1 + k_2 + k_3 + k_4 + k_5 + k_6\)
\item
  \(\displaystyle \prod_{i=1}^{i=6}k_i = k_1 \cdot k_2 \cdot k_3 \cdot k_4 \cdot k_5 \cdot k_6\)
\item
  \(\displaystyle \sum_{i=1}^{i=3}i^k = 1^k + 2^k + 3^k\)
\item
  \(\displaystyle \prod_{i=1}^{i=3}i^k = 1^k \cdot 2^k \cdot 3^k\)
\item
  \(\displaystyle \sum_{i=1}^{n}i = 1 + 2 + 3 + ... + n\) we are using dots (\ldots) to represent all the number until \(n\). Here, thanks to Gauss we can also write \(\displaystyle \sum_{i=1}^{n}i = \frac{n(n+1)}{2}\), i.e.~Gauss formula for sum of first \(n\) natural numbers
\end{enumerate}

Exr. \ref{exr:m-notations-sigma-product-reverse}

\begin{enumerate}
\def\labelenumi{\alph{enumi})}
\item
  \(1+2+3+4+5+6 = \displaystyle \sum_{k=1}^{6}k\)
\item
  \(2^2+3^2+4^2+5^2 = \displaystyle \sum_{x=2}^{5}x^2\)
\item
  \(4 \cdot 5 \cdot 6 \cdot 7 \cdot 8 = \displaystyle \prod_{x=4}^{8}x\)
\item
  \(1 + \frac{1}{2} + \frac{1}{3} + \frac{1}{4} + \frac{1}{5} + ... + \frac{1}{n} = \displaystyle \sum_{k=1}^{n}\frac{1}{k}\)
\end{enumerate}

\hypertarget{sets}{%
\chapter{Sets}\label{sets}}

\textbf{Aims}

\begin{itemize}
\tightlist
\item
  to introduce sets and basic operations on sets
\end{itemize}

\textbf{Learning outcomes}

\begin{itemize}
\tightlist
\item
  to be able to explain what a set is
\item
  to be able to construct new sets from given sets using the basic set operations
\item
  to be able to use Venn diagrams to shows all possible logical relations between two and three sets
\end{itemize}

\hypertarget{definitions}{%
\section{Definitions}\label{definitions}}

\begin{itemize}
\tightlist
\item
  \textbf{set}: a well-defined collection of distinct objects, e.g.~\(S = \{2, 4, 6\}\)
\item
  \textbf{elements}: the objects that make up the set are also known as \textbf{elements} of the set
\item
  if \(x\) is an element of \(S\), we say that \(x\) belongs to \(S\) and write \(x \in S\) and if \(x\) is not an element of \(S\) we say that \(x\) does not belong to \(S\) and write \(x \notin S\)
\item
  a set may contain \textbf{finitely} many or \textbf{infinitely} many elements
\end{itemize}

\begin{itemize}
\tightlist
\item
  \textbf{subset, \(\subseteq\)}: if every element of set A is also in B, then A is said to be a subset of B, written as \(A \subseteq B\) and pronounced A is contained in B, e.g.~\(A \subseteq B\), when \(A = \{2, 4, 6\}\) and \$ = \(B = \{2, 4, 6, 8, 10\}\). Every set is a subset of itself.
\item
  \textbf{superset}: for our outs \(A\) and \(B\) we can also say that \(B\) is a \textbf{superset} of \(A\) and write \(B \supset A\)
\end{itemize}

\begin{itemize}
\tightlist
\item
  \textbf{cardinality}: the number of elements within a set \(S\), denoted as \(|S|\)
\item
  \textbf{empty set, \(\emptyset\)}: is a unique set with no members, denoted by \(E = \emptyset\) or \(E = \{\}\). The empty set is a subset of every set.
\end{itemize}

\hypertarget{basic-set-operations}{%
\section{Basic set operations}\label{basic-set-operations}}

\begin{itemize}
\tightlist
\item
  \textbf{union of two sets, \(\cup\) }: two sets can be ``added'' together, the union of A and B, written as \(A \cup B\), e.g.~\(\{1, 2\} \cup \{2, 3\} = \{1, 2, 3\}\) or \(\{1, 2, 3\} \cup \{1, 4, 5, 6\} = \{1, 2, 3, 4, 5, 6\}\)
\item
  \textbf{intersection of two sets, \(\cap\)}: a new set can be constructed by taking members of two sets that are ``in common'', written as \(A \cap B\), e.g.~\(\{1, 2, 3, 4, 5, 6\} \cap \{2, 3, 7\} = \{2, 3\}\) or \(\{1, 2, 3\} \cap \{7 \} = \{\emptyset \}\)
\end{itemize}

\begin{itemize}
\tightlist
\item
  \textbf{complement of a set, \(A'\), \(A^c\)}: are the elements not in A
\item
  \textbf{difference of two sets, \(\setminus\)}: two sets can be ``subtracted'', denoted by \(A \setminus B\), by taking all elements that are members of A but are not members of B, e.g.~\(\{1, 2, 3, 4\} \setminus \{1, 3\} = \{2, 4\}\). This is also in other words a relative complement of A with respect to B.
\end{itemize}

\begin{itemize}
\tightlist
\item
  \textbf{partition of a set}: a partition of a set S is a set of nonempty subset of S, such that every element x in S is in exactly one of these subsets. That is, the subset are pairwise \emph{disjoint}, meaning no two sets of the partition contain elements in common, and the union of all the subset of the partition is S, e.g.~Set \(\{1, 2, 3\}\) has five partitions: i) \(\{1\}, \{2\}, \{3\}\), ii) \(\{1, 2\}, \{3\}\), iii) \(\{1,3\}, \{2\}\), iv) \(\{1\}, \{2, 3\}\) and v) \(\{1,2,3\}\)
\end{itemize}

\hypertarget{venn-diagrams}{%
\section{Venn diagrams}\label{venn-diagrams}}

Venn diagram is a diagram that shows all possible logical relations between a finite collection of different sets. A Venn diagram shows elements as points in the plane, and sets as regions inside closed curves. A Venn diagram consists of multiple overlapping closed curves, usually circles, each representing a set.

E.g. given \(A = \{1, 2, 5\}\) and \(B = \{1, 6\}\) Venn diagram of \(A\) and \(B\):

\begin{center}\includegraphics{102-math-sets_files/figure-latex/unnamed-chunk-2-1} \end{center}

And given \(A = \{1, 2, 5\}\), \(B = \{1, 6\}\) and \(C= \{4, 7\}\) Venn diagram of \(A\), \(B\) and \(C\):

\begin{center}\includegraphics{102-math-sets_files/figure-latex/unnamed-chunk-3-1} \end{center}

And given \(A = \{1, 2, 3, 4, 5, 6\}\) and \(B= \{2, 4, 6\}\) Venn diagram of \(A\) and \(B\):

\begin{center}\includegraphics{102-math-sets_files/figure-latex/unnamed-chunk-4-1} \end{center}

\begin{center}\rule{0.5\linewidth}{0.5pt}\end{center}

\hypertarget{exercises-sets}{%
\section{Exercises: sets}\label{exercises-sets}}

\begin{exercise}
\protect\hypertarget{exr:m-sets-01}{}{\label{exr:m-sets-01} }
Given set \(S = \{1, 2, 3, 4, 5, 6\}\):

\begin{enumerate}
\def\labelenumi{\alph{enumi})}
\tightlist
\item
  what is the subset \(T\) of \(S\) consisting of its even elements?
\item
  what is the complement \(T^c\)?
\item
  what is the subset \(U\) of \(S\) containing of the prime numbers in \(S\)?
\item
  what is the intersection \(T \cap U\)?
\item
  what is the union of \(T \cup U\)?
\item
  what is the set difference \(U \setminus T\)?
\end{enumerate}
\end{exercise}

\begin{exercise}
\protect\hypertarget{exr:m-sets-02}{}{\label{exr:m-sets-02} }
Given set \[A = \{cat, elephant, dog, turtle, goldfish, hamster, parrot, tiger, guinea pig, lion\}\]

\begin{enumerate}
\def\labelenumi{\alph{enumi})}
\tightlist
\item
  what is the subset \(D\) of \(A\) consiting of domesticated animals?
\item
  what is the subset \(C\) of \(A\) consiting of Felidae (cat) family?
\item
  what is the intersection of \(D\) and \(C\)?
\item
  what is the complement of \(D\), \(D^c\)?
\item
  what is the union of \(D\) and \(C\)?
\item
  what is the set difference of \(A \setminus C\)?
\item
  can you draw Venn diagram showing relationship between \(D\) and \(C\)?
\end{enumerate}
\end{exercise}

\hypertarget{answers-to-selected-exercises-sets}{%
\section*{Answers to selected exercises (sets)}\label{answers-to-selected-exercises-sets}}
\addcontentsline{toc}{section}{Answers to selected exercises (sets)}

Exr. \ref{exr:m-sets-01}

\begin{enumerate}
\def\labelenumi{\alph{enumi})}
\tightlist
\item
  \(T = \{2, 4, 6\}\)
\item
  \(T^c = \{1, 3, 5\}\), i.e.~\(T^c\) contains all the elements of \(S\) not in \(T\)
\item
  \(U = \{2, 3, 5\}\), the primes in \(S\)
\item
  \(T \cap U = \{2\}\), common elements of \(T\) and \(U\), i.e.~the even and prime numbers
\item
  \(T \cup U = \{2, 3, 4, 5, 6\}\)
\item
  \(U \setminus T = \{3, 5\}\), consisting of the elements of \(U\) that are not in \(T\)
\end{enumerate}

\hypertarget{functions}{%
\chapter{Functions}\label{functions}}

\textbf{Aims}

\begin{itemize}
\tightlist
\item
  to revisit the concept of a function
\end{itemize}

\textbf{Learning outcomes}

\begin{itemize}
\tightlist
\item
  to be able to explain what function, function domain and function range are
\item
  to be able to identify input, output, argument, independent variable, dependent variable
\item
  to be able to evaluate function for a given value and plot the function
\end{itemize}

\hypertarget{definitions-1}{%
\section{Definitions}\label{definitions-1}}

\begin{figure}

{\centering \includegraphics[width=30.88in]{figures/precourse/math-functions-definition} 

}

\caption{Formal function defition}\label{fig:func-def}
\end{figure}

\begin{itemize}
\tightlist
\item
  A \textbf{function}, \(f(x)\), can be viewed as a rule that relates input \(x\) to an output \(f(x)\)
\item
  In order for a rule to be a function it must produce a single output for any given input
\item
  Input \(x\) is also known as \textbf{argument} of the function
\item
  \textbf{Domain of a function}: the set of all values that the function ``maps''
\item
  \textbf{Range}: the set of all values that the function maps into
\end{itemize}

\textbf{Many names are used interchangeably}

Functions have been around for a while and there are many alternative names and writing conventions are being used. Common terms worth knowing:

\begin{figure}

{\centering \includegraphics[width=31.19in]{figures/precourse/math-functions-terms} 

}

\caption{Common function terms}\label{fig:func-ters}
\end{figure}

\hypertarget{evaluating-function}{%
\section{Evaluating function}\label{evaluating-function}}

To evaluate a function is to replace (substitute) its variable with a given number or expression. E.g. given a rule (function) that maps temperature measurements from Celsius to Fahrenheit scale:
\[f(x) = 1.8x + 32\]
where \(x\) is temperature measurements in Celsius and \(f(x)\) is the associated value in Fahrenheit, we can find for a given temperature in Celsius corresponding temperature in Fahrenheit. Let's say we measure 10 Celsius degrees one autumn day in Uppsala and we want to share this information with a friend in USA. We can find the equivalent temperature in Fahrenheit by evaluating our function at 10, \(f(10)\), giving us \[f(10) = 1.8\cdot 10 + 32 = 50\]

\hypertarget{plotting-function}{%
\section{Plotting function}\label{plotting-function}}

Function graphs are a convenient way of showing functions, by looking at the graph it is easier to notice function's properties, e.g.~for which input values functions yields positive outcomes or whether the function is increasing or decreasing. To graph a function, one can start by evaluating function at different values for the argument \(x\) obtaining \(f(x)\), plotting the points by plotting the pairs \((x, f(x))\) and connecting the dots. E.g. evaluating our above temperature rule at -20, -10, 0, 10, 20, 30 Celsius degrees results in:

\begin{longtable}[]{@{}ccc@{}}
\toprule()
x (Celsius degrees) & evaluates & f(x) (Fahrenheit degrees) \\
\midrule()
\endhead
-20 & \(f(-20) = 1.8 \cdot (-20) + 32\) & -4 \\
-10 & \(f(-10) = 1.8 \cdot (-10) + 32\) & 14 \\
0 & \(f(0) = 1.8 \cdot (0) + 32\) & 32 \\
10 & \(f(10) = 1.8 \cdot (10) + 32\) & 50 \\
20 & \(f(20) = 1.8 \cdot (20) + 32\) & 68 \\
30 & \(f(30) = 1.8 \cdot (30) + 32\) & 86 \\
\bottomrule()
\end{longtable}

\begin{figure}

{\centering \includegraphics{103-math-functions_files/figure-latex/unnamed-chunk-2-1} 

}

\caption{Graph of f(x) for the temeprature rule}\label{fig:unnamed-chunk-2}
\end{figure}

\hypertarget{standard-classes-of-functions}{%
\section{Standard classes of functions}\label{standard-classes-of-functions}}

\textbf{Algebraic function}: functions that can be expressed as the solution of a polynomial equation with integer coefficients, e.g.~

\begin{itemize}
\tightlist
\item
  constant function \(f(x) = a\)
\item
  identity function \(f(x) = x\)
\item
  linear function \(f(x) = ax + b\)
\item
  quadratic function \(f(x) = a + bx + cx^2\)
\item
  cubic function \(fx() = a + bx + cx^2 + dx^3\)
\end{itemize}

\textbf{Transcendental functions}: functions that are not algebraic, e.g.~

\begin{itemize}
\tightlist
\item
  exponential function \(f(x) = e^x\)
\item
  logarithmic function \(f(x) = log(x)\)
\item
  trigonometric function \(f(x) = -3sin(2x)\)
\end{itemize}

\begin{figure}

{\centering \includegraphics{103-math-functions_files/figure-latex/unnamed-chunk-3-1} 

}

\caption{Examples of the standard classess of functions}\label{fig:unnamed-chunk-3}
\end{figure}

\hypertarget{piecewise-functions}{%
\section{Piecewise functions}\label{piecewise-functions}}

A function can be in pieces, i.e.~we can create functions that behave differently based on the input \(x\) value. They are useful to describe situations in which a rule changes as the input value crosses certain ``boundaries''. E.g. a function value could be fixed in a given range and equal to the input value (identify function) for input values outside this range

\begin{equation}
    f(x) =
    \left\{
        \begin{array}{cc}
                2 & \mathrm{if\ } x \le 1 \\
                x & \mathrm{if\ } x>1 \\
        \end{array}
    \right.
\end{equation}

The function can be split in many pieces, e.g.~the personal training fee in SEK may depend whether the personal trainer is hired for an hour, two hours or three or more hours:
\begin{equation}
    f(h) =
    \left\{
        \begin{array}{cc}
                500  & \mathrm{if\ } h \le 1 \\
                750  & \mathrm{if\ } 1 < h \le 2 \\
                500 + 250 \cdot h & \mathrm{if\ } h > 2 \\
        \end{array}
    \right.
\end{equation}

\begin{figure}

{\centering \includegraphics{103-math-functions_files/figure-latex/unnamed-chunk-4-1} 

}

\caption{Examples of piece-wise functions}\label{fig:unnamed-chunk-4}
\end{figure}

\begin{center}\rule{0.5\linewidth}{0.5pt}\end{center}

\hypertarget{exercises-functions}{%
\section{Exercises: functions}\label{exercises-functions}}

\begin{exercise}
\protect\hypertarget{exr:m-functions-evaluate-01}{}{\label{exr:m-functions-evaluate-01} }
Given the function for the personal trainer costs:

\begin{equation}
    f(h) =
    \left\{
        \begin{array}{cc}
                500  & \mathrm{if\ } h \le 1 \\
                750  & \mathrm{if\ } 1 < h \le 2 \\
                500 + 250 \cdot h & \mathrm{if\ } h > 2 \\
        \end{array}
    \right.
\end{equation}

How much would you pay

\begin{enumerate}
\def\labelenumi{\alph{enumi})}
\tightlist
\item
  for a 4-hours session? Evaluate function f(h) for value 4.
\item
  for a 2-hour session? Evalue function f(h) for value 2.
\end{enumerate}
\end{exercise}

\begin{exercise}
\protect\hypertarget{exr:m-functions-write}{}{\label{exr:m-functions-write} }
A museum charges 50 SEK per person for a guided tour with a group of 1 to 9 people or a fixed 500 SEK fee for a group of 10 or more people. Write a function relating the number of people \(n\) to the cost \(C\).
\end{exercise}

\begin{exercise}
\protect\hypertarget{exr:m-functions-plot-evaluate}{}{\label{exr:m-functions-plot-evaluate} }
Given function

\begin{equation}
    f(x) =
    \left\{
        \begin{array}{cc}
                x^2  & \mathrm{if\ } x \le 1 \\
                3  & \mathrm{if\ } 1 < x \le 2 \\
                x & \mathrm{if\ } x > 2 \\
        \end{array}
    \right.
\end{equation}

\begin{enumerate}
\def\labelenumi{\alph{enumi})}
\tightlist
\item
  sketch a graph of a function for \(x \in (-4, 4)\), i.e.. for \(x\) between -4 and 4
\item
  evaluate function at f(1)
\item
  evaluate function at f(4)
\end{enumerate}
\end{exercise}

\hypertarget{answers-to-selected-exercises-functions}{%
\section*{Answers to selected exercises (functions)}\label{answers-to-selected-exercises-functions}}
\addcontentsline{toc}{section}{Answers to selected exercises (functions)}

Exr. \ref{exr:m-functions-evaluate-01}

\begin{enumerate}
\def\labelenumi{\alph{enumi})}
\tightlist
\item
  \(f(4) = 500 + 250 \cdot 4 = 1500\)
\item
  \(f(2) = 750\) as \(h \le 2\) means less or equal to 2, that is including 2
\end{enumerate}

\hypertarget{differentiation}{%
\chapter{Differentiation}\label{differentiation}}

\textbf{Aims}

\begin{itemize}
\tightlist
\item
  introduce the concept of differentiation and rules of differentiation
\end{itemize}

\textbf{Learning outcomes}

\begin{itemize}
\tightlist
\item
  to be able to explain differentiation in terms of rate of change
\item
  to be able to find derivatives in simple cases
\end{itemize}

\hypertarget{rate-of-change}{%
\section{Rate of change}\label{rate-of-change}}

\begin{itemize}
\tightlist
\item
  We are often interested in the rate at which some variable is changing, e.g.~we may be interested in the rate at which the temperature is changing or the rate of water levels increasing
\item
  Rapid or unusual changes may indicate that we are dealing with unusual situations, e.g.~global warming or a flood
\item
  Rates of change can be positive, negative or zero corresponding to a variable increasing, decreasing and non-changing
\end{itemize}

\begin{figure}

{\centering \includegraphics{104-math-differentiation_files/figure-latex/diff-rate-1} 

}

\caption{The function $f(x)$ changes at different rates for different values of $x$}\label{fig:diff-rate}
\end{figure}

The function \(f(x) = x^4 - 4x^3 - x^2 - e^{-x}\) changes at different rates for different values of \(x\), e.g.~

\begin{itemize}
\tightlist
\item
  between \(x \in (-10, -9)\) the \(f(x)\) is increasing at slightly higher pace than \(x \in (5,6)\)
\item
  between \(x \in (-7, -5)\) the \(f(x)\) is decreasing and
\item
  between \(x \in (0, 1)\) the \(f(x)\) is not changing
\item
  to be able to talk more precisely about the rate of change than just saying ``large and positive'' or ``small and negative'' change we need to quantify the changes, i.e.~assign the rate of change an exact value
\item
  \textbf{Differentiation} is a technique for calculating the rate of change of any function
\end{itemize}

\hypertarget{average-rate-of-change-across-an-interval}{%
\section{Average rate of change across an interval}\label{average-rate-of-change-across-an-interval}}

\begin{figure}

{\centering \includegraphics[width=13.42in]{figures/precourse/math-differentiation-01} 

}

\caption{The average rate of change of $f(x)$ with respect to $x$ over $[a, b]$ is equal to the slope of the secant line (in black)}\label{fig:diff-01}
\end{figure}

To dive further into calculating the rate of change let's look at Figure \ref{fig:diff-01} and define the \emph{average rate of change} of a function across an interval. Figure \ref{fig:diff-01} shows a function \(f(x)\) with two possible argument values \(a\) and \(b\) marked and their corresponding function values \(f(a)\) and \(f(b)\).

Consider that \(x\) is increasing from \(a\) to \(b\). The change in \(x\) is \(b-a\), i.e.~as \(x\) increases from \(a\) to \(b\) the function \(f(x)\) increase from \(f(a)\) to \(f(b)\). The change in \(f(x)\) is \(f(b)-f(a)\) and the average rate of change of \(y\) across the \([a,b]\) interval is:

\begin{equation}
\frac{change\:in\:y}{change\:in\:x}=\frac{f(b)-f(a)}{b-a}
\label{eq:diff-point}
\end{equation}

E.g. let's take a quadratic function \(f(x)=x^2\) and calculate the average rate of change across the interval \([1, 4]\).

\begin{itemize}
\tightlist
\item
  The change in \(x\) is \(4-1\) and the change in \(f(x)\) is \(f(4) - f(1) = 4^2 -1^2 = 16 - 1 = 15\). So the average rate of change is \(\frac{15}{3}=5\). What does this mean? It means that across the interval \([1,4]\) on average the \(f(x)\) value increases by 5 for every 1 unit increase in \(x\).
\item
  If we were to look at the average rate of change across the interval \([-2, 0]\) we would get \(\frac{f(0)-f(-2)}{0 - (-2)}=\frac{0 - (-2)^2}{2}=\frac{-4}{2} = -2\). Here, over the \([-2, 0]\) on average the \(f(x)\) value decreases by 2 for every 1 unit increase in \(x\).
\item
  Looking at the graph of \(f(x)=x^2\) verifies our calculations
\end{itemize}

\begin{figure}

{\centering \includegraphics{104-math-differentiation_files/figure-latex/diff-avg-rate-1} 

}

\caption{Example function $f(x) = x^2$}\label{fig:diff-avg-rate}
\end{figure}

\hypertarget{rate-of-change-at-a-point}{%
\section{Rate of change at a point}\label{rate-of-change-at-a-point}}

\begin{itemize}
\tightlist
\item
  We often need to know the rate of change of a function at a point, and not simply an average rate of change across an interval.
\item
  Figure \ref{fig:diff-02}, similar to Figure \ref{fig:diff-01}, shows, instead of two points \(a\) and \(b\), point \(a\) and a second point defined in terms of its distance from the first point \(a\). Thus, the two points are now \(a\) and \(a + h\) and the distance between the two points is equal to \(h\).
\item
  Now we can write that:
  \[\frac{change\:in\:y}{change\:in\:x}=\frac{f(a+h)-f(a)}{a+h-a} = \frac{f(a+h)-f(a)}{h}\]
\end{itemize}

\begin{figure}

{\centering \includegraphics[width=13.06in]{figures/precourse/math-differentiation-02} 

}

\caption{The average rate of change of $f(x)$ with respect to $x$ over $[a, b]$ is equal to the slope of the secant line (in black)}\label{fig:diff-02}
\end{figure}

Further:

\begin{itemize}
\tightlist
\item
  if we assume that the second point \(a+h\) is really close to \(a\), meaning that \(h\) approaches 0, denoted as \(h \rightarrow 0\), we can find the rate of change at the point \(a\)
\item
  the distance between the two points \(a\) and \(a+h\) is getting smaller and so is the difference of the function values \(f(a+h) - f(a)\). We denote these small differences as \(\delta x\) and \(\delta y\), pronounced ``delta x'' and ``delta y'', respectively.
\item
  the term \(\delta\) reads as ``delta'' and represents a small change
\end{itemize}

We can thus continue and write that a \textbf{rate of change of a function at a point} is given by
\begin{equation}
\frac{small\:change\:in\:y}{small\:change\:in\:x} = \lim_{h\to0}\frac{f(a+h)-f(a)}{h}
\label{eq:diff-point-2}
\end{equation}

E.g. let's look at the linear function \(f(x) = 2x+3\). We can find the rate of change at any point of \(x\) by:
\[\frac{small\:change\:in\:y}{small\:change\:in\:x} = \\\frac{f(x+h)-f(x)}{x+h-x}= \lim_{h\to0}\frac{2(x+h)+3-(2x+3)}{x+h-x}=\lim_{h\to0}\frac{2h}{h}=2\]
It means that the function value \(f(x)\) increases by 2 for every small increase, \(h\), in \(x\). Here, this increase is the same for all the values of \(x\), i.e.~it does not depend on \(x\). \textbf{The change in function value \(f(x)\) can depend} on the value of \(x\), for instance if we look at the quadratic \(f(x)=x^2\) function, we get:
\[\frac{small\:change\:in\:y}{small\:change\:in\:x} = \\ \frac{f(x+h)-f(x)}{x+h-x}=\lim_{h\to0}\frac{x^2+2xh+h^2-x^2}{h}=\lim_{h\to0}\frac{2xh+h^2}{h}=2x+h\]
This means that:

\begin{itemize}
\tightlist
\item
  the rate of change for the function \(f(x)\) at a point \(x\) is \(2x\)
\item
  the \(f(x)\) value increases by \(2x\) for every small increase, \(h\), in \(x\)
\item
  the rate of change along a quadratic function is changing constantly according to the value of \(x\) we are looking at, it is a function of \(x\)
\item
  and finally that the rate of change does not give us any information about the rate of change globally.
\end{itemize}

\hypertarget{terminology-and-notation}{%
\section{Terminology and notation}\label{terminology-and-notation}}

\begin{itemize}
\tightlist
\item
  \textbf{differentiation} is the process of finding the rate of change of a given function
\item
  the function is said to be \textbf{differentiated}
\item
  the rate of change of a function is also known as the \textbf{derivative} of the function
\item
  given a function \(f(x)\) we say that we differentiate function in respect to \(x\) and write:
\end{itemize}

\[\lim_{h\to0}\frac{\delta y}{\delta x}= \frac{dy}{dx}\]

or use the ``prime'' \[f´(x)\]

\hypertarget{table-of-derivatives}{%
\section{Table of derivatives}\label{table-of-derivatives}}

\begin{itemize}
\tightlist
\item
  in practice, there is no need to compute \(\displaystyle \lim_{h\to0}\frac{\delta y}{\delta x}\) every time when we want to find a derivative of a function
\item
  instead, we can use patterns of the common functions and their derivatives
\end{itemize}

\begin{longtable}[]{@{}cc@{}}
\caption{\label{tab:diff-table} Common functions and their derivatives, \(k\) denotes a constant}\tabularnewline
\toprule()
Function \(f(x)\) & Derivative \(f'(x)\) \\
\midrule()
\endfirsthead
\toprule()
Function \(f(x)\) & Derivative \(f'(x)\) \\
\midrule()
\endhead
\(k\) & \(0\) \\
\(x\) & \(1\) \\
\(kx\) & \(k\) \\
\(x^n\) & \(nx^{n-1}\) \\
\(kx^n\) & \(knx^{n-1}\) \\
\(e^x\) & \(e^x\) \\
\(e^{kx}\) & \(ke^{kx}\) \\
\(\ln(x)\) & \(\frac{1}{x}\) \\
\(\ln(kx)\) & \(\frac{1}{x}\) \\
\bottomrule()
\end{longtable}

We can use the Table \ref{tab:diff-table} to find derivatives of some of the functions e.g.

\begin{itemize}
\tightlist
\item
  \(f(x) = 3x\), \(f'(x) = 3\)
\item
  \(f(x) = 2x^4\), \(f'(x) = 2*4x^{4-1} = 8x^3\)
\item
  \(f(x) = e^{2x}\), \(f'(x) = 2e^{2x}\)
\item
  \(f(x) = \ln(4x)\), \(f'(x) = \frac{1}{x}\)
\end{itemize}

\hypertarget{exercises-differentiation}{%
\section{Exercises: differentiation}\label{exercises-differentiation}}

\begin{exercise}
\protect\hypertarget{exr:m-diff}{}{\label{exr:m-diff} }
Find derivatives of the functions

\begin{enumerate}
\def\labelenumi{\alph{enumi})}
\tightlist
\item
  \(f(x) = 2\)
\item
  \(f(x) = 2x + 1\)
\item
  \(f(x) = 5x^2\)
\item
  \(f(x) = 4x^3 + x^2\)
\item
  \(f(x) = \sqrt(x)\)
\item
  \(f(x) = \ln(2x)\)
\item
  \(f(x) = e^{x}\)
\item
  \(f(x) = \frac{9}{x^2} + ln(4x)\)
\item
  \(f(x) = 4x−6x^6\)
\item
  \(f(x) = \frac{3}{x^2}\)
\end{enumerate}
\end{exercise}

\hypertarget{answers-to-selected-exercises-differentiation}{%
\section*{Answers to selected exercises (differentiation)}\label{answers-to-selected-exercises-differentiation}}
\addcontentsline{toc}{section}{Answers to selected exercises (differentiation)}

Exr. \ref{exr:m-diff}

\begin{enumerate}
\def\labelenumi{\alph{enumi})}
\tightlist
\item
  \(f(x) = 2\), \(f'(x) = 0\)
\item
  \(f(x) = 2x + 1\), \(f'(x) = 2\)
\item
  \(f(x) = 5x^2\), \(f'(x)= 10x\)
\item
  \(f(x) = 4x^3 + x^2\), \(f'(x)=12x^2 + 2x\)
\item
  \(f(x) = \sqrt(x) = x^{\frac{1}{2}}\), \(f'(x)=\frac{1}{2}x^{\frac{1}{2}-1} = \frac{1}{2}x^{-\frac{1}{2}}\)
\item
  \(f(x) = \ln(2x)\), \(f'(x) = \frac{1}{x}\)
\item
  \(f(x) = e^{x}\), \(f'(x) = e^x\)
\end{enumerate}

\hypertarget{integration}{%
\chapter{Integration}\label{integration}}

\textbf{Aims}

\begin{itemize}
\tightlist
\item
  to introduce the concept of integration
\end{itemize}

\textbf{Learning outcomes}

\begin{itemize}
\tightlist
\item
  to be able to explain what integration is
\item
  to be able to explain the relationship between differentiation and integration
\item
  to be able to integrate simple functions
\item
  to to able to use integration to calculate the area under the curve in simple cases
\end{itemize}

\hypertarget{reverse-to-differentiation}{%
\section{Reverse to differentiation}\label{reverse-to-differentiation}}

\begin{itemize}
\tightlist
\item
  when a function \(f(x)\) is known we can differentiate it to obtain the derivative \(f'(x)\)
\item
  the reverse process is to obtain \(f(x)\) from the derivative
\item
  this process is called \textbf{integration}
\item
  apart from simple reversing differentiation integration comes very useful in finding \textbf{areas under curves}, i.e.~the area above the x-axis and below the graph of \(f(x)\), assuming that \(f(x)\) is positive
\item
  the symbol for integration is \(\int\) and is known as ``integral sign''
\end{itemize}

E.g. let's take a function \(f(x) = x^2\). Suppose we only have a derivative, which is \(f'(x) = 2x\) and we would like to find the function given this derivative. Formally we write: \[\int 2x dx = x^2 +c\]

where:

\begin{itemize}
\tightlist
\item
  the term \(2x\) within the integral is called the \textbf{integrand}
\item
  the term \(dx\) indicates the name of the variable involved, here \(x\)
\item
  \(c\) is \textbf{constant of integration}
\end{itemize}

\hypertarget{what-is-constant-of-integration}{%
\section{What is constant of integration?}\label{what-is-constant-of-integration}}

\begin{itemize}
\tightlist
\item
  Integration reverses the process of differentiation, here, given our example function \(f(x) = x^2\) that we pretended we do not know, we started with the derivative \(f´(x) = 2x\) and via integration we obtained back the very function \[\int 2x dx = x^2\]
\item
  However, many function can result in the very same derivative since the derivative of a constant is 0 e.g.~a derivatives of \(f(x) = x^2\), \(f(x) = x^2 + 10\) and \(f(x) = x^2 + \frac{1}{2}\) all equal to \(f'(x) = 2x\)
\item
  We have to take this into account when we are integrating, i.e.~reverting differentiation. As we have no way of knowing what the original function constant is, we add it in form of \(c\), i.e.~unknown constant, called the constant of integration.
\end{itemize}

\hypertarget{table-of-integrals}{%
\section{Table of integrals}\label{table-of-integrals}}

Similar to differentiation, in practice we can use tables of integrals to be able to find integrals in simple cases

\begin{longtable}[]{@{}cc@{}}
\caption{\label{tab:int-table} Common functions and their integrals, \(k\) denotes a constant}\tabularnewline
\toprule()
Function \(f(x)\) & Integral \(\int f(x) dx\) \\
\midrule()
\endfirsthead
\toprule()
Function \(f(x)\) & Integral \(\int f(x) dx\) \\
\midrule()
\endhead
\(constant\:k\) & \(kx + c\) \\
\(x\) & \(\frac{x^2}{2}+c\) \\
\(kx\) & \(k\frac{x^2}{2}+c\) \\
\(x^n\) & \(\frac{x^{n+1}}{n+1}+c\;\; if\;n\neq-1\) \\
\(kx^n\) & \(k\frac{x^{n+1}}{n+1}+c\) \\
\(e^x\) & \(e^x+c\) \\
\(e^{kx}\) & \(\frac{e^{kx}}{k}+c\) \\
\(\frac{1}{x}\) & \(\ln(x)+c\) \\
\bottomrule()
\end{longtable}

E.g.

\begin{itemize}
\tightlist
\item
  \(\int 4x^3 dx = \frac{4x^{3+1}}{3+1}=x^4 + c\)
\item
  \(\int (x^2 + x) dx = \frac{x^3}{3} + \frac{x^2}{2} +c\) (note: we can evaluate integrals separately and add them as integration as differentiation is linear)
\end{itemize}

\hypertarget{definite-integrals}{%
\section{Definite integrals}\label{definite-integrals}}

\begin{itemize}
\item
  the above examples of integrals are \textbf{indefinite integrals}, the result of finding an indefinite integral is usually a function plus a constant of integration
\item
  we have also \textbf{definite integrals}, so called because the result is a definite answer, usually a number, with no constant of integration
\item
  definite integrals are often used to areas bounded by curves or, as we will cover later on, estimating probabilities
\item
  we write: \[\int_{a}^bf(x)dx\] where:
\item
  \(\int_{a}^bf(x)dx\) is called the definite integral of \(f(x)\) from \(a\) to \(b\)
\item
  the numbers \(a\) and \(b\) are known as lower and upper limits of the integral
\end{itemize}

E.g. let's look at the function \(f(x) = x\) plotted below and calculate a definite integral from \(0\) to \(2\).

\begin{figure}

{\centering \includegraphics{105-math-integration_files/figure-latex/int-area-1} 

}

\caption{Graph of function $f(x) = x$}\label{fig:int-area}
\end{figure}

We write \[\int_{0}^2f(x)dx = \int_{0}^2 xdx =  \Bigr[ \frac{1}{2}x^2\Bigr]_0^2 = \frac{1}{2}(2)^2 - \frac{1}{2}(0)^2 = 2\] so first find the integral and then we evaluate it at upper limit and subtracting the evaluation at the lower limit. Here, the result it 2. What would be the result if you tried to calculate the triangle area on the above plot, area defined by the blue vertical lines drawn at 0 and 2 and horizontal x-axis? The formula for the triangle area is \(Area = \frac{1}{2}\cdot base \cdot height\) so here \(Area = \frac{1}{2} \cdot 2 \cdot 2 = 2\) the same result as achieved with integration.

\begin{center}\rule{0.5\linewidth}{0.5pt}\end{center}

\hypertarget{exercises-integration}{%
\section{Exercises: integration}\label{exercises-integration}}

\begin{exercise}
\protect\hypertarget{exr:m-int}{}{\label{exr:m-int} }
Integrate:

\begin{enumerate}
\def\labelenumi{\alph{enumi})}
\tightlist
\item
  \(\int 2 \cdot dx\)
\item
  \(\int 2x\cdot dx\)
\item
  \(\int (x^4 + x^2 + 1)\cdot dx\)
\item
  \(\int e^x\cdot dx\)
\item
  \(\int e^{2x}\cdot dx\)
\item
  \(\int \frac{2}{x}\cdot dx\)
\item
  \(\int_2^4 2x\cdot dx\)
\item
  \(\int_0^4 (x^2+1)dx\)
\item
  \(\int (x^4 + \frac{2}{x} + e^{2x}) dx\)
\item
  \(\int_0^4 (x^4+1) dx\)
\end{enumerate}
\end{exercise}

\hypertarget{answers-to-selected-exercises-integration}{%
\section*{Answers to selected exercises (integration)}\label{answers-to-selected-exercises-integration}}
\addcontentsline{toc}{section}{Answers to selected exercises (integration)}

Exr. \ref{exr:m-int}

\begin{enumerate}
\def\labelenumi{\alph{enumi})}
\tightlist
\item
  \(\int 2 \cdot dx = 2x +c\)
\item
  \(\int 2x\cdot dx = \frac{2x^2}{2} = x^2 + c\)
\item
  \(\int (x^4 + x^2 + 1)\cdot dx = \frac{x^5}{5} + \frac{x^3}{3} + x + c\)
\item
  \(\int e^x\cdot dx = e^x + c\)
\item
  \(\int e^{2x}\cdot dx = \frac{1}{2}e^{2x}\)
\item
  \(\int \frac{2}{x}\cdot dx =\int 2\cdot \frac{1}{x}\cdot dx = 2 \ln{x}+ c\)
\item
  \(\int_2^4 2x\cdot dx = \Bigr[x^2\Bigr]_2^4 = 16 - 4 = 12\)
\item
  \(\int_0^4 (x^2+1)dx = \Bigr[\frac{x^3}{3} + x \Bigr]_0^4=\frac{4^3}{3}+4 - 0 = \frac{64}{3}+4 = \frac{76}{3}\)
\end{enumerate}

\hypertarget{vectors}{%
\chapter{Vectors}\label{vectors}}

\textbf{Aims}

\begin{itemize}
\tightlist
\item
  to introduce vectors and basic vectors operations
\end{itemize}

\textbf{Learning outcomes}

\begin{itemize}
\tightlist
\item
  to be able to write \(n\)-dimensional vectors using vector notations
\item
  to be able to perform addition and scalar multiplication
\item
  to be able to check if two vectors are orthogonal
\end{itemize}

A large number of statistical models use vectors and matrices, both for compact representations, and for the calculations, e.g.~parameter estimates.

\hypertarget{vectors-1}{%
\section{Vectors}\label{vectors-1}}

\begin{itemize}
\tightlist
\item
  A vector is an ordered set of number
\item
  These numbers, e.g.~in vector \(\mathbf{x}\) can be expressed as a row \(\mathbf{x}=[6\quad 0\quad 5 \dots1]\)
\item
  or as a column \(\mathbf{x}=\begin{bmatrix} 6 \\ 0 \\ 5 \\ \vdots \\ 1 \end{bmatrix}\)
\item
  the number of elements in a vector is referred to as its \textbf{dimension} and we often use \(n\) to express \(n\)-dimensional vector, where \(n\) can be any natural number
\item
  here, we denote vectors using small bold font \(\mathbf{x}\), other notations may include an arrow \(\vec x\) or overline \(\overline{x}\)
\item
  also \textbf{parentheses} are used interchangeably with \textbf{square bracket}, e.g.~\(\mathbf{x}=[6\quad 0\quad 5 \dots1]\) can be written as \(\mathbf{x}=(6\quad 0\quad 5 \dots1)\) or \(\begin{pmatrix} 6\\ 0\\ 5\\ \vdots \\ 1 \end{pmatrix}\)
\end{itemize}

\hypertarget{operations-on-vectors}{%
\section{Operations on vectors}\label{operations-on-vectors}}

Given two vectors of the same dimension:
\(\mathbf{x}=\begin{bmatrix}  x_1 \\  x_2 \\  x_3 \\  \vdots \\  x_n \end{bmatrix}\)
and
\(\mathbf{y}=\begin{bmatrix}  y_1 \\  y_2 \\  y_3 \\  \vdots \\  y_n \end{bmatrix}\)

\textbf{Addition}: we add two vectors, element by element \(\mathbf{x} + \mathbf{y}=\begin{bmatrix}  x_1 + y_1 \\  x_2 + y_2 \\  x_3 + y_3 \\  \vdots \\  x_n + y_n \end{bmatrix}\)

\textbf{Scalar multiplication}: we can multiple vector by a numerical value, scalar, denoted as \(\gamma\):
\[\gamma \cdot \mathbf{x} =\begin{bmatrix}
  \gamma \cdot x_1 \\ 
  \gamma \cdot x_2 \\
  \gamma \cdot x_3 \\
  \vdots \\
  \gamma \cdot x_n 
\end{bmatrix}\]

\textbf{Difference} \(\mathbf{x} - \mathbf{y}\) can be written as \(\mathbf{x} + (-1) \cdot \mathbf{y}\), thus we multiply second vector with \(-1\) and then add two vectors

\textbf{Linear combination of vectors}: the vector \(\gamma \cdot \mathbf{x} + \delta \cdot \mathbf{y}\) is said to be a linear combination of \(\mathbf{x}\) and \(\mathbf{y}\):
\[\gamma \cdot \mathbf{x} + \delta \cdot \mathbf{y} =\begin{bmatrix}
  \gamma \cdot x_1 + \delta \cdot y_1 \\ 
  \gamma \cdot x_2 + \delta \cdot y_2\\
  \gamma \cdot x_3 + \delta \cdot y_3\\
  \vdots \\
  \gamma \cdot x_n + \delta \cdot y_n
\end{bmatrix}\]

\textbf{Inner product of vectors} is given by: \[\mathbf{x} \cdot \mathbf{y} = x_1 \cdot y_1 + x_2 \cdot y_2 + \dots x_n \cdot y_n = \displaystyle\sum_{i=1}^{n}x_i\cdot y_i\]

\textbf{Orthogonality of vectors}: two vectors are said to be orthogonal if their inner product is zero \[\mathbf{x} \cdot \mathbf{y} =\displaystyle\sum_{i=1}^{n}x_i\cdot y_i = 0\]

\hypertarget{null-and-unit-vector}{%
\section{Null and unit vector}\label{null-and-unit-vector}}

\begin{itemize}
\tightlist
\item
  a \textbf{null vector} is a vector whose elements are all \(0\); the difference between any vector and itself yields a null vector
\item
  a \textbf{unit vector} is a vector whose elements are all \(1\)
\end{itemize}

\begin{center}\rule{0.5\linewidth}{0.5pt}\end{center}

\hypertarget{exercises-vectors}{%
\section{Exercises: vectors}\label{exercises-vectors}}

\begin{exercise}
\protect\hypertarget{exr:m-vectors}{}{\label{exr:m-vectors} }Based on vector definitions and operations:

\begin{enumerate}
\def\labelenumi{\alph{enumi})}
\item
  find the vector \(\mathbf{x} + \mathbf{y}\) when \(\mathbf{x} =\begin{bmatrix}  1 \\  2 \\  5 \end{bmatrix}\) and \(\mathbf{y} =\begin{bmatrix}  0\\  3 \\  1  \end{bmatrix}\)
\item
  find the vector \(2\mathbf{x} - \mathbf{y}\) when \(\mathbf{x} =\begin{bmatrix}  -2 \\  3 \\  5 \end{bmatrix}\) and \(\mathbf{y} =\begin{bmatrix}  0\\  -4 \\  7  \end{bmatrix}\)
\item
  are \(\mathbf{u}\) and \(\mathbf{v}\) vectors orthogonal when when \(\mathbf{u} =\begin{bmatrix}  1 \\  2 \end{bmatrix}\) and \(\mathbf{v} =\begin{bmatrix}  2\\  -1  \end{bmatrix}\)?
\item
  are \(\mathbf{u}\) and \(\mathbf{v}\) vectors orthogonal when when \(\mathbf{u} =\begin{bmatrix}  3 \\  -1 \end{bmatrix}\) and \(\mathbf{v} =\begin{bmatrix}  7\\  5  \end{bmatrix}\)?
\item
  find the value \(n\) such that the vectors \(\mathbf{u} =\begin{bmatrix}  2 \\  4 \\  1 \end{bmatrix}\) and \(\mathbf{v} =\begin{bmatrix}  n\\  1 \\  8  \end{bmatrix}\) are orthogonal.
\end{enumerate}
\end{exercise}

\hypertarget{answers-to-selected-exercises-vectors-and-matrices}{%
\section*{Answers to selected exercises (vectors and matrices)}\label{answers-to-selected-exercises-vectors-and-matrices}}
\addcontentsline{toc}{section}{Answers to selected exercises (vectors and matrices)}

Exr. \ref{exr:m-vectors}

\begin{enumerate}
\def\labelenumi{\alph{enumi})}
\tightlist
\item
\end{enumerate}

\[\mathbf{x} + \mathbf{y} = \begin{bmatrix} 1 \\ 2 \\ 5 \end{bmatrix} + \begin{bmatrix} 0 \\ 3 \\ 1 \end{bmatrix} = \begin{bmatrix} 1 + 0\\ 2 + 3 \\ 5 + 1 \end{bmatrix} = \begin{bmatrix} 1 \\ 5 \\ 6 \end{bmatrix}\]

\begin{enumerate}
\def\labelenumi{\alph{enumi})}
\setcounter{enumi}{1}
\item
  \[2\mathbf{x} - \mathbf{y} = \begin{bmatrix} 2 \cdot (-2) \\ 2 \cdot 3 \\ 2 \cdot 5 \end{bmatrix} + \begin{bmatrix} (-1) \cdot 0 \\ (-1) \cdot (-4) \\ (-1) \cdot 7 \end{bmatrix} = \begin{bmatrix} -4 \\ 6 \\ 10 \end{bmatrix}  + \begin{bmatrix} 0 \\ 4 \\ -7 \end{bmatrix}  = \begin{bmatrix} -4 + 0 \\ 6 + 4 \\ 10 - 7 \end{bmatrix} = \begin{bmatrix} -4 \\ 10 \\ 3 \end{bmatrix}\]
\item
  Yes, to check orthogonality we need to calculate the inner product of two vectors and see if it is equal to 0, here
  \(\mathbf{u} \cdot \mathbf{v} =\displaystyle\sum_{i=1}^{2}u_i\cdot v_i = 1 \cdot 2 + 2 \cdot (-1) = 2 - 2 = 0\)
\item
  No, since the inner product does not equal to 0 \[\mathbf{u} \cdot \mathbf{v} =\displaystyle\sum_{i=1}^{2}u_i\cdot v_i = 3 \cdot 7 + (-1) \cdot 5 = 21 - 5 = 16 \neq 0\]
\end{enumerate}

\hypertarget{matrices}{%
\chapter{Matrices}\label{matrices}}

\textbf{Aims}

\begin{itemize}
\tightlist
\item
  to introduce matrix and basic matrices operations
\end{itemize}

\textbf{Learning outcomes}

\begin{itemize}
\tightlist
\item
  to be able to write matrices using matrix notations
\item
  to be able to perform simple matrix operations such as adding and multiplication
\item
  to be able to find the reverse of the 2-dimensional matrix
\end{itemize}

\hypertarget{matrix}{%
\section{Matrix}\label{matrix}}

A matrix is a rectangular array of numbers e.g.~

\[\mathbf{A}=\begin{bmatrix}
  x_{11} & x_{12} & x_{13} & \dots & x_{1n} \\
  x_{21} & x_{22} & x_{23} & \dots & x_{2n} \\
  \dots & \dots & \dots& \dots & \dots\\
  x_{m1} & x_{m2} & x_{1m3} & \dots & x_{mn} \\
\end{bmatrix}\]

where:

\begin{itemize}
\tightlist
\item
  the notional subscripts in the typical element \(x_{ij}\) refers to its row and column location in the array, e.g.~\(x_{12}\) refers to element in the first row and second column
\item
  we say that matrix has \(m\) rows and \(n\) columns and the \textbf{dimension} of a matrix is defined as \(m \times n\)
\item
  a matrix can be viewed as a set of column vectors or a set of row vectors
\item
  a vector can be viewed as a matrix with only one column or with only one row
\end{itemize}

\hypertarget{special-matrices}{%
\section{Special matrices}\label{special-matrices}}

\begin{itemize}
\tightlist
\item
  A matrix with the same number of rows as columns, \(m = n\), is said to be a \textbf{square matrix}
\item
  A matrix that is not squared, \(m \neq n\) is called \textbf{rectangular matrix}
\item
  A \textbf{null matrix} is composed of all 0
\item
  An \textbf{identity matrix}, denoted as \(\mathbf{I}\) or \(\mathbf{I_n}\), is a square matrix with 1's on the main diagonal and all other elements equal to 0, e.g.~a three-dimensional identity matrix is \[\mathbf{I}=\begin{bmatrix}
  1 & 0 & 0  \\
  0 & 1 & 0  \\
  0 & 0 & 1
  \end{bmatrix}\]
\item
  A square matrix is said to be \textbf{symmetric} if \(x_{ij} = x_{ji}\) e.g.~
  \[\mathbf{A}=\begin{bmatrix}
  1 & 4 & 2  \\
  4 & 1 & 0  \\
  2 & 0 & 1
  \end{bmatrix}\]
\item
  A \textbf{diagonal matrix} is a square matrix whose non-diagonal entries are all zero, that is \(x_{ij} = 0\) for \(i \neq j\), e.g.~
  \[\mathbf{A}=\begin{bmatrix}
  1 & 0 & 0  \\
  0 & 2 & 0  \\
  0 & 0 & 3
  \end{bmatrix}\]
\item
  An \textbf{upper-triangular matrix} is a square matrix in which all entries below the diagonal are 0, that is \(x_{ij}=0\) for \(i<j\) e.g.~
  \[\mathbf{A}=\begin{bmatrix}
  1 & 3 & 4  \\
  0 & 2 & 5  \\
  0 & 0 & 3
  \end{bmatrix}\]
\item
  A \textbf{lower-triangular matrix} is a square matrix in which all entries above the diagonal are 0, that is hat is \(x_{ij}=0\) for \(i>j\) e.g.~
  \[\mathbf{A}=\begin{bmatrix}
  1 & 0 & 0  \\
  1 & 1 & 0  \\
  1 & 1 & 1
  \end{bmatrix}\]
\end{itemize}

\hypertarget{matrix-operations}{%
\section{Matrix operations}\label{matrix-operations}}

\begin{itemize}
\tightlist
\item
  matrix \(\mathbf{A} = \mathbf{B}\) if both matrices have exactly the same dimension and if each element of \(\mathbf{A}\) equals to the corresponding element of e.g.~\(\mathbf{A} = \mathbf{B}\) if
  \(\mathbf{A}=\begin{bmatrix} 1 & 3 & 4 \\ 0 & 2 & 5 \\ 0 & 0 & 3 \end{bmatrix}\) and \(\mathbf{B}=\begin{bmatrix} 1 & 3 & 4 \\ 0 & 2 & 5 \\ 0 & 0 & 3 \end{bmatrix}\)
\end{itemize}

\begin{itemize}
\tightlist
\item
  for any matrix \(\mathbf{A}\) the \textbf{transpose}, denoted by \(\mathbf{A}^\top\) or \(\mathbf{A}^\prime\), is obtained by interchanging rows and columns, e.g.~given matrix \(\mathbf{A}=\begin{bmatrix} 1 & 3 & 4 \\ 0 & 2 & 5 \\ 0 & 0 & 3 \end{bmatrix}\) we have \(\mathbf{A}^\top=\begin{bmatrix} 1 & 0 & 0 \\ 3 & 2 & 0 \\ 4 & 5 & 3 \end{bmatrix}\). The transpose of a transpose of a matrix yield the original matrix, \(\Big(\mathbf{A}^\top\Big)^\top = \mathbf{A}\)
\end{itemize}

\begin{itemize}
\tightlist
\item
  we can \textbf{add} two matrices if they have the same dimension, e.g.~
  \[\mathbf{A} + \mathbf{B} = \mathbf{A} =\begin{bmatrix}
  x_{11} & x_{12}   \\
  x_{21} & x_{22} 
  \end{bmatrix} + \begin{bmatrix}
  y_{11} & y_{12}   \\
  y_{21} & y_{22} 
  \end{bmatrix} = \begin{bmatrix}
  x_{11}+y_{11} & x_{12}+y_{12}   \\
  x_{21}+y_{21} & x_{22}+y_{22} 
  \end{bmatrix}\]
\end{itemize}

\begin{itemize}
\tightlist
\item
  we can \textbf{multiply} a matrix by \textbf{a scalar} \(\delta\) e.g.~\[\delta \cdot \mathbf{A} = \begin{bmatrix}
  \delta \cdot x_{11} & \delta \cdot x_{12}   \\
  \delta \cdot x_{21} & \delta \cdot x_{22} 
  \end{bmatrix}\]
\end{itemize}

\begin{itemize}
\tightlist
\item
  we can \textbf{multiply two matrices} if they are \textbf{conformable}, i.e.~first matrix has the same number of columns as the number of rows in the second matrix. We then can write:
  \[\mathbf{C} = \mathbf{A} \cdot \mathbf{B}  = \begin{bmatrix}
  x_{11} & x_{12} & x_{13}  \\
  x_{21} & x_{22} & x_{23}
  \end{bmatrix} \times \begin{bmatrix}
  y_{11} & y_{12}   \\
  y_{21} & y_{22}  \\
  y_{31} & y_{32}
  \end{bmatrix} = \\\ 
  \begin{bmatrix}
  x_{11} \cdot y_{11} + x_{12} \cdot y_{21} + x_{13} \cdot y_{31}  & x_{11} \cdot y_{12} + x_{12} \cdot y_{22} + x_{13} \cdot y_{32}  \\
  x_{21} \cdot y_{11} + x_{22} \cdot y_{21} + x_{23} \cdot y_{31} & x_{21} \cdot y_{12} + x_{22} \cdot y_{22} + x_{23} \cdot y_{32}
  \end{bmatrix}\]
\end{itemize}

\hypertarget{inverse-of-a-matrix}{%
\section{Inverse of a matrix}\label{inverse-of-a-matrix}}

For a square matrix \(\mathbf{A}\) there may exist a matrix \(\mathbf{B}\) such that \(\mathbf{A} \cdot \mathbf{B} = \mathbf{I}\). An \textbf{inverse}, if it exists, is denoted as \(\mathbf{A}^{-1}\) and we can rewrite the definition as \[\mathbf{A} \cdot \mathbf{A}^{-1} = \mathbf{I}\] where \(\mathbf{I}\) is an identify matrix (equivalent to 1). There is no division for matrices, instead we can use inverse to multiply the matrix by an inverse, similar to when instead of dividing the number \(a\) by \(b\) we multiply \(a\) by reciprocal of \(b = \frac{1}{b}\)

For a 2-dimensional matrix we can follow the below formula for obtaining the inverse
\[\begin{bmatrix}
  x_{11} & x_{12}   \\
  x_{21} & x_{22} 
\end{bmatrix}^{-1} = \frac{1}{x_{11} \cdot x_{22} - x_{12} \cdot x_{21}} \cdot \begin{bmatrix}
  x_{22} & -x_{12}   \\
  -x_{21} & x_{11} 
\end{bmatrix}\]

\hypertarget{orthogonal-matrix}{%
\section{Orthogonal matrix}\label{orthogonal-matrix}}

\begin{itemize}
\tightlist
\item
  A matrix \(\mathbf{A}\) for which \(\mathbf{A^\top} = \mathbf{A^{-1}}\) is true is said to be \textbf{orthogonal}
\end{itemize}

\begin{center}\rule{0.5\linewidth}{0.5pt}\end{center}

\hypertarget{exercises-matrices}{%
\section{Exercises: matrices}\label{exercises-matrices}}

\begin{exercise}
\protect\hypertarget{exr:m-matrix}{}{\label{exr:m-matrix} }
Given matrices

\(\mathbf{A} = \begin{bmatrix}  1 & 2 \\  3 & 4  \end{bmatrix}\),
\(\mathbf{B} = \begin{bmatrix}  1 & 0 \\  0 & 1  \end{bmatrix}\) and \(\mathbf{C} = \begin{bmatrix}  1 & 0 \\  0 & 2  \end{bmatrix}\)

\begin{enumerate}
\def\labelenumi{\alph{enumi})}
\tightlist
\item
  what is the dimension of matrix \(\mathbf{A}\)?
\item
  what is \(\mathbf{A}^\top\)?
\item
  which of the matrices is i) an identity matrix ii) a square matrix, iii) null matrix, iv) diagonal matrix, v) a triangular matrix,?
\item
  calculate \(\mathbf{A} + \mathbf{B}\)?
\item
  calculate \(\mathbf{A} \cdot \mathbf{C}\)?
\item
  calculate \(\mathbf{B}^\top\)
\item
  calculate \(\mathbf{A}^{-1}\)
\item
  calculate \((\mathbf{A} + \mathbf{B})^{-1}\)
\item
  answer again a) - h) this time using R functions and/or commands
\end{enumerate}
\end{exercise}

\hypertarget{answers-to-selected-exercises-matrices}{%
\section*{Answers to selected exercises (matrices)}\label{answers-to-selected-exercises-matrices}}
\addcontentsline{toc}{section}{Answers to selected exercises (matrices)}

Exr. \ref{exr:m-matrix}

\begin{enumerate}
\def\labelenumi{\alph{enumi})}
\item
  \(2 \times 2\)
\item
  \(\mathbf{A}^\top = \begin{bmatrix}  1 & 3 \\  2 & 4  \end{bmatrix}\)
\item
  \begin{enumerate}
  \def\labelenumii{\roman{enumii})}
  \tightlist
  \item
    identity matrix: \(\mathbf{B}\), ii) a square matrix: \(\mathbf{A}\), \(\mathbf{B}\) and \(\mathbf{C}\), iii) null matrix: none, iv) diagonal matrix: \(\mathbf{B}\) (identity matrix is diagonal) and \(\mathbf{C}\), v) triangular \(\mathbf{B}\) and \(\mathbf{C}\) as both identify matrix \(\mathbf{B}\) and diagonal matrix \(\mathbf{C}\) is triangular, both lower and upper triangular
  \end{enumerate}
\item
  \(\mathbf{A} + \mathbf{B} = \begin{bmatrix}  1 & 2 \\  3 & 4  \end{bmatrix} + \begin{bmatrix}  1 & 0 \\  0 & 1  \end{bmatrix} = \begin{bmatrix}  2 & 2 \\  3 & 5  \end{bmatrix}\)
\item
  \(\mathbf{A} \cdot \mathbf{C} = \begin{bmatrix}  1 \cdot 1 + 2 \cdot 0 & 1 \cdot 0 + 2 \cdot 2 \\  3 \cdot 1 + 4 \cdot 0 & 3 \cdot 0 + 4 \cdot 2  \end{bmatrix} = \begin{bmatrix}  1 & 4 \\  3 & 8  \end{bmatrix}\)
\item
  \(\mathbf{B}^\top = \begin{bmatrix}  1 & 0 \\  0 & 1  \end{bmatrix}\)
\item
\end{enumerate}

\[\mathbf{A}^{-1} = \begin{bmatrix}
  1 & 2   \\
  3 & 4 
\end{bmatrix}^{-1} = \frac{1}{1 \cdot 4 - 2 \cdot 3} \cdot \begin{bmatrix}
  4 & -2   \\
  -3 & 1
\end{bmatrix} = -\frac{1}{2} \cdot \begin{bmatrix}
  4 & -2   \\
  -3 & 1
\end{bmatrix} = \begin{bmatrix}
  -2 & 1   \\
  \frac{3}{2} & -\frac{1}{2}
\end{bmatrix}\]

\end{document}
